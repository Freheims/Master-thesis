% !TEX encoding = UTF-8 Unicode
%!TEX root = ../Main/thesis.tex
% !TEX spellcheck = en-US
%%=========================================
\documentclass[../Main/thesis.tex]{subfiles}
\begin{document}
\chapter{Conclusion}
\label{ch:conclusion}
This chapter presents a summary of the research presented in this thesis, limitations of the research project, and suggestions on what could be done in the future.

\section{Conclusions}
Design Science Research was used to develop FireTracker, a system for tracking firefighters while exercising smoke diving.
The system was developed through three iterations, and the prototype created during those iterations was evaluated.
This prototype consisted of three components: an Android application for collecting data, a back-end for processing data, and an exercise management tool for creating and presenting data.
The two first components are presented in this thesis, but all of them are interrelated and was evaluated together.

In the first iteration the requirements for the system was established. 
A wireframe for the Android application was created, together with a visual demonstration of how the app could work.
This was presented to the fire department, and their feedback on it, and their ideas and requirements for the system combined with observations made during a smoke was used to establish requirements for the system.

The second iteration started with the development of the first functional prototype of the system.
In this iteration the most of the tracking functionality was implemented in the Android application.
The back-end was developed with web-server functionality and the first version of the localization algorithm.
At the end of this iteration the system was demonstrated for, and tested by, a smoke diving instructor at Øygarden Fire and Rescue.


In the third iteration the feedback from iteration two was used to improve the prototype.
The localization algorithm in the back-end was improved, and the Android app got a visual overhaul.
During this iteration technical tests of the system and the BLE beacons were performed, and the results were used in the development of the final prototype.

The system was evaluated by testing it in a smoke diving exercise together with ØFR.
Two instructors and two smoke divers participated in the testing.
After the tests the two instructors and smoke divers were interviewed using semi-structured interviews, and they answered a System Usability Scale Questionnaire.

This research shows that Bluetooth Low Energy Beacons have potential to be used for tracking firefighters during smoke diving, but there are still some issues with the technology that needs to be resolved before a reliable system can be created and used.
And in its current state the data gathered from the beacons are not precise enough to be used as feedback in a smoke diving training setting.

\section{Future Work}
If more work is to be done on this project in the future, or on similar projects the research should focus on either finding ways to restrict the beacon's signal strength or develop a new localization algorithm that handles this problem better.
When these issues are resolved the possibility for the instructor to see the tracking in real-time would be a desired feature from the fire department.
Research on how to track firefighters in a real-life emergency situation would be interesting both for use in training, and if the tracking is in real-time, it could be used to enhance the safety of the firefighters.

\end{document}
