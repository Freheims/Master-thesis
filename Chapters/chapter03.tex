% !TEX encoding = UTF-8 Unicode
%!TEX root = ../Main/thesis.tex
% !TEX spellcheck = en-US
%%=========================================
\documentclass[../Main/thesis.tex]{subfiles}
\begin{document}
\chapter{Research methodology}
\label{ch:research_methodology}
This chapter describes the research methodologies, methods and frameworks used in this research project.
The research addresses the two questions established in Chapter~\ref{ch:reserch_questions}:

\begin{enumerate}
	\item \textbf{How can BLE beacons be used to support firefighter movement tracking under smoke diving training?}
	\item \textbf{Do they provide good enough information for useful feedback?}
\end{enumerate}

\section{Design Science}
Design Science research is a research methodology that establishes and uses research when the goal is an artifact or a recommendation.
Research based on design science can be performed in an organizational context or an academic environment \citep{lacerda2015design}.
As the goal of this project is to create and evaluate an artifact, the design science methodology fits well.

\citet{hevner2004design} claims that design is both a process and a artifact.
The design process produces an innovative artifact, which is evaluated to provide feedback about the artifact itself, and a better understanding of the problem. 
This information is used to improve both the artifact itself and the design process.

Design Science can be seen as a problem solving process.
The foundation on which design science is based is that knowledge about and understanding of a design problem, and the solution to that problem, is something you achieve through the development and use of an artifact.
Artifacts as a result of design science is according to \citet{March1995} \textit{constructs}, \textit{models}, \textit{methods}, and \textit{instantiations}.

%\todo{Add more about how and why I use design science}

\subsection{Artifacts}
Within design science artifacts are created to address unsolved problems, and when they are evaluated by looking at the utility they provide in solving the problem.
\citet[p.78-79]{hevner2004design} identify those four artifacts as follows:

\begin{enumerate}
	\item \textit{Constructs}: provides the language, vocabulary and concepts, which are used to define and communicate the problems and solutions.
	\item \textit{Models}: are representations of a real world situation of the design problem and its solution space created using constructs. They help with the understanding of the problem and the solution, and is often used to represent the connection between them. 
	\item \textit{Methods}: provide guidance on how to search the solution space and solve problems.
	\item \textit{Instantiations}: is an implementation of a construct, model or method in a working system.
\end{enumerate}

In the study reported here the main artifact is the back end of the FireTracker system, which determines the movements and activity of the fire fighters based on the collected data and it is therefore an instantiation.

\subsection{Guidelines}
\citet{hevner2004design} defines a set of seven guidelines for implementing and evaluating design science research.
Those guidelines and their definitions are listed in table~\ref{tab:design-science-guidelines}.
%The following sections describe how those guidelines apply to this research.

\begin{table}[ht]
\centering
\begin{tabular}{|p{0.4\linewidth}|p{0.6\linewidth}|}
	\hline
	\textbf{Guideline} & \textbf{Description} \\ \hline
	Guideline 1: Design as an Artifact & Design-science research must produce a viable artifact in the form of a construct, a model, a method, or an instantiation. \\ \hline
	Guideline 2: Problem relevance & The objective of design-science research is to develop technology-based solutions to important and relevant business problems. \\ \hline
	Guideline 3: Design evaluation & The utility, quality, and efficacy of a design artifact must be rigorously demonstrated via well-executed evaluation methods. \\ \hline
	Guideline 4: Research contributions & Effective design-science research must provide clear and verifiable contributions in the areas of the design artifact, design foundations, and/or design methodologies. \\ \hline
	Guideline 5: Research rigor & Design-science research relies upon the application of rigorous methods in both the construction and evaluation of the design artifact. \\ \hline
	Guideline 6: Design as a search process & The search for an effective artifact requires utilizing available means to reach desired ends while satisfying laws in the problem environment. \\ \hline
	Guideline 7: Communication of research & Design-science research must be presented effectively both to technology-oriented as well as management-oriented audiences. \\ \hline
\end{tabular}
\caption{Design Science Research Guidelines \citep[p.83]{hevner2004design}}
\label{tab:design-science-guidelines}
\end{table}

%\subsubsection*{Guideline 1: Design as an Artifact}
%The result of this research project will be a functioning 
%\subsubsection*{Guideline 2: Problem relevance}
%There are currently no available technology for 
%\subsubsection*{Guideline 3: Design evaluation}
%
%\subsubsection*{Guideline 4: Research contributions}
%\subsubsection*{Guideline 5: Research rigor}
%\subsubsection*{Guideline 6: Design as a search process}
%\subsubsection*{Guideline 7: Communication of research}


\section{Prototyping}
A prototype is a early version of a product. 
It is supposed to demonstrate some features or properties of an artifact which can be evaluated and analyzed to gain some understanding about how the product can be revised in the next iteration of the development \citep{oates2005researching}.
There are two types of prototypes: \textit{low-fidelity-prototype} and \textit{high-fidelity-prototype}
The goal of prototyping is to get user feedback.

Low-fidelity prototyping is a way of translating high-level designs into testable artifacts \citep{Babich2017}.
The purpose of a low-fidelity prototype is to test and check functionality, not visual designs and appearances.
In a low-fidelity prototype only the most important elements of the content is included, and only general visual attributes are presented.

High-fidelity prototypes look and function as similar to the actual product as possible \citep{Babich2017}.
They are usually created when there is a solid understanding of the final product and it needs to be tested on actual users.
In a high-fidelity prototype most of the content is included, and the design is realistic with all details in place.

On this project the goal was to create a High-fidelity prototype with the tracking-functionality working. 

%\subsection{Requirements}
%\subsection{Iterative development}
%\subsection{Testing}

\section{Data Collection}
Collecting data is an important part of the research process, and therefore is choosing the right method for doing so also important.
This section presents the two methods used in this project: interviews and experiments.

\subsection{Interview}
Interviews are one of the most common techniques for collecting qualitative data.
There are three types of qualitative interviews: unstructured interview, semi-structured interview, and individual in-depth interview \citep{DiCicco-Bloom2006}.
The interviews in this project was conducted as semi-structured interviews.

A semi-structured interview combines the features of unstructured and structured interviews and use both open and closed questions \citep{Preece2011}.
The general way of performing a semi-structured interview is to have a set of open questions which is used as guidance, and with open answers to inquire as much information as possible.
From those open questions other new questions can emerge during the interview \citep{DiCicco-Bloom2006}.

\subsection{Experiments}



\end{document}
