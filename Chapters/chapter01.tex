% !TEX encoding = UTF-8 Unicode
%!TEX root = ../Main/thesis.tex
% !TEX spellcheck = en-US
%%=========================================
\documentclass[../Main/thesis.tex]{subfiles}
\begin{document}
\chapter{Introduction}
\label{ch:introduction}

The organizations Uni Research Health, Centre for the Science of Learning \& Technology (SLATE), Western Norway University of Applied Sciences (HVL), ENOVATE AS and Øygarden Brann og Redning (Øygarden fire and rescue) are currently cooperating on the project ``Inquire Competence for Better Practice and Assessment'' (iComPAss).
This project does, among other things, study how one could use data driven decision-making to improve the training of firefighters.
As a part of that study they want to research how and if Bluetooth or similar technologies could be used to track the movements of firefighters when they exercise smoke diving.
 
Today we have pretty good systems for outdoor localization, positioning and tracking using the Global Positioning System (GPS). 
GPS is available almost everywhere and because most smartphones has a GPS-antenna it is easy to use it in your own application. 
Unfortunately GPS does not work well enough indoors when you need precise locations.
There are however developed several other projects for indoor localization and tracking using different technologies such as light \citep{xiaohan2010improved}, sound \citep{schweinzer2010ultrasonic}, WIFI  \citep{chang2010robust} and Bluetooth \todo{Find reference}, but none of those systems are commonly available or in widespread use yet.


\section{Motivation}



\section{Research Question}




\onlyinsubfile{\bibliographystyle{chicago}}
\onlyinsubfile{\bibliography{../library}}
\end{document}
