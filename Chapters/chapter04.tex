% !TEX encoding = UTF-8 Unicode
%!TEX root = ../Main/thesis.tex
% !TEX spellcheck = en-US
%%=========================================
\documentclass[../Main/thesis.tex]{subfiles}
\begin{document}
\chapter{Technologies}
\label{ch:technologies}

Several different technologies have been used for this project, and they can be divided into three main categories depending on what part of the system they belongs to: app, back end, and front end.
This chapter is an overview of the technologies we used in our project, and an explanation of how and why we used them.
It will also describe some alternative technologies we could have used.
The intention is not to give a comprehensive presentation of the technologies, but instead give a brief introduction to support our choices.

\section{Android}
Android is the most used operating system for smartphones \citep{osmarketshare}. 
It is developed by google and is based on the Linux kernel.
Android was initially released in 2005 \citep{Morrill2008a} and the current version is Android 8.1 ``Oreo'' \citep{Burke2017a}.

The alternative to using Android for our project would be to use Apples iOS.
Since both me and my co-student have experience developing for Android and neither of us have developed for iOS earlier we chose to develop for Android as it would be time-saving to use technology we were already familiar with.
Another reason for choosing Android is that we both have Android devices available for testing, but only one of us owns an iOS device.
It is also recommended to use a Mac when developing for iOS, which again we do not have available.

\subsection{Android Beacon Library}
Android Beacon Library (ABL) is an Android library that allows Android devices to interact with BLE beacons. 
It simplifies the process of searching for beacons and retrieve information, such as id, name and RSSI from them.
The library can also be used to listen for beacons in the background \citep{RadiusNetwork2015}.
We decided to use this library because it is one of few libraries that is actively maintained.
It is also compatible with several different beacon-standards, and does not require proprietary beacons such as the Estimote SDK \citep{Estimote2017}.
This is an advantage because we are not limited to using only one beacon manufacturer or beacon type.

\subsection{Retrofit}
\subsection{Eventbus}


\section{Back end}
\subsection{Gin}
\subsection{GORM}
\subsection{SQLite}


\section{Front end}
\subsection{React}

\section{Programming Languages}
\subsection{Golang}
\subsection{Java}
\subsection{JavaScript}

\section{JSON}

\section{REST}

\section{Linux}


\section{Alternative Technologies}
\subsection{Pozyx}
\subsection{Arduino}
\subsection{Estimote}



\onlyinsubfile{\bibliographystyle{chicago}}
\onlyinsubfile{\bibliography{../library}}
\end{document}
